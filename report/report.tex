\documentclass[10pt]{report}

\usepackage[utf8]{inputenc}
\usepackage[T1]{fontenc}
\usepackage[english]{babel}
\usepackage[left=2cm,top=1.5cm,right=2cm,bottom=2cm]{geometry}

\usepackage{mathtools}
\usepackage{amsfonts}
\usepackage{amssymb}
\usepackage{amsopn}
\usepackage{amsthm}
\usepackage[hidelinks]{hyperref}
\usepackage{cleveref}
\usepackage{fancyvrb}
\usepackage{tikz}
\usepackage[justification=centering]{caption}
% \patchcmd{\thebibliography}{\chapter*}{\section*}{}{}
\usepackage[super]{nth}
\usepackage{textcomp}
\usepackage{enumitem}
\usepackage{doi}
\setlist{nosep}

\newtheorem{theorem}{Theorem}[section]
\newtheorem*{theorem*}{Theorem}
\newtheorem{lemma}[theorem]{Lemma}
\newtheorem{cor}[theorem]{Corollary}

\setcounter{secnumdepth}{3}
\setcounter{tocdepth}{1}
%\renewcommand\thesection{\arabic{section}}

\newcommand{\R}{\ensuremath{\mathbb{R}}}
\newcommand{\Rb}{\ensuremath{\overline{\mathbb{R}}}}
\newcommand{\N}{\ensuremath{\mathbb{N}}}
\newcommand{\Q}{\ensuremath{\mathbb{Q}}}
\newcommand{\Z}{\ensuremath{\mathbb{Z}}}
\newcommand{\C}{\ensuremath{\mathbb{C}}}
\newcommand{\U}{\ensuremath{\mathbb{U}}}
\newcommand{\F}{\ensuremath{\mathbb{F}}}
\newcommand{\K}{\ensuremath{\mathbb{K}}}


\newcommand\eqdef{\stackrel{\mathclap{\mbox{\tiny def}}}{=}}




% \usepackage{titlesec}
% \titleformat{\chapter}[hang] 

\makeatletter
\def\@makechapterhead#1{%
  \vspace*{0\p@}% %%% removed!
  {\parindent \z@ \raggedright \normalfont
    \ifnum \c@secnumdepth >\m@ne
        \huge\bfseries \@chapapp\space \thechapter
        \par\nobreak
        \vskip 10\p@
    \fi
    \interlinepenalty\@M
    \Huge \bfseries #1\par\nobreak
    \vskip 30\p@
  }}
\def\@makeschapterhead#1{%
  \vspace*{0\p@}% %%% removed!
  {\parindent \z@ \raggedright
    \normalfont
    \interlinepenalty\@M
    \Huge \bfseries  #1\par\nobreak
    \vskip 30\p@
  }}
\makeatother

\usepackage{titlesec}
\titlespacing*{\section}{0pt}{.7\baselineskip}{.5\baselineskip}
\titlespacing*{\subsection}{0pt}{.4\baselineskip}{.3\baselineskip}

\begin{document}

\begin{titlepage}
  \centering
  {\scshape\huge École Normale Supérieure \par}
  \vspace{0.3cm}
  {\scshape\Large Department of Mathematics and their Applications (DMA) \par}
  \vspace{3cm}
  {\Huge\bfseries Error estimation in maxlike reconstruction for quantum tomography \par}
  \vspace{0.5cm}
  {\scshape\Large Internship report\par}
  \vspace{3cm}
  {\LARGE Thibaut \textsc{Pérami}\par}
  \vfill
  {
    \large
    supervised by\par
    Igor \textsc{Dotsenko}\par
    and\par
    Pierre \textsc{Rouchon}\par
  in\par
  LKB, Collège de France
  }

  \vfill

  % Bottom of the page
  {\Large \today\par}
\end{titlepage}

\newcommand{\fset}{\ensuremath{\mathop{\text{\textquotesingle}}}}


% \pagenumbering{gobble} % to avoid counting contents page in report
\tableofcontents

\chapter*{Introduction}
% \pagenumbering{arabic}
\addcontentsline{toc}{chapter}{Introduction}

This report details my internship with Pierre Rouchon and Igor dotsenko on
Quantum tomography by maximum-likelihood reconstruction. It was done mainly at
the Laboratory Kessler-Brosel (LKB) in the Collège de France but also a bit
\emph{insert Pierre's location here}.

The goal of Quantum tomography is to find ways to reconstruct the state of
a quantum system from several direct or indirect measurement on it. In some case
those measurement do not modify the state of the system. Those are called QND
(Quantum Non-Destructive) measurement.

The specific physical project on which I worked with Igor Dotsenko is the
internship project of Luis Najera about thermodynamics in the case of
atom-cavity interaction in quantum optics. More precisely, we send an circular
Rydberg atom in a resonant cavity containing photons. The state of the cavity is
thermal i.e it follow the Bose-Einstein distribution. The atom then interacts
with the cavity. If the atom is in a thermal state between two atomic level that
are at the same frequency than the cavity then a thermal exchange will occur in
the expected way. However if we pump one of the state of the atom in a hidden
state, like a Maxwell demon, we can can apparently break the second law and make
a cold atom give head to a hotter cavity.

In order to study this experiment, we need to measure the state of the cavity.
That state cannot be measured directly, so we do it by quantum tomography using
maximum-likelihood reconstruction. This construction takes into account all the
measurements made and deduce a density matrix. To find it, one needs to solve a
convex optimisation problem on the set of density matrix (hermitian positive
definite matrix of trace 1). This was done using a gradient descent method with
projection of the gradient on the domain.

The deduced state is usually not a pure state because the reconstruction often
came from a state that has been prepared several times.
All the prepared state are slightly different.
Furthermore, other factors like measurements imperfection and decoherent
relaxation decrease the precision of the reconstruction.

With this measurements, we can now get and interpret the results of our Maxwell
demon experiment. However, to be able to interpret measurement, we need to know
the error on the reconstructed state. The original paper [ref here] on
tomography only provide the standard deviation of usual quantum operator which
are linear in the density matrix. However to do our analysis of the second law,
we need to evaluate the error on the entropy.

In order to do that, I had to extract from the reconstruction the probability
distribution around the maximum which is our estimator. But this distribution
cannot be samples easily because its a mix of Gaussian and exponential law
truncated into the domain of density matrix. Into to sample from this
probability I use an adapted hit and run method to build a Markov chain on the
density matrix space (but discrete time) whose stationary distribution is the
wanted distribution. By sampling this chain long enough, I can get the statistic
average and standard deviation of any real function of the density matrix space.

In the end we get nice plots with errorbars and we were able to determine which
point we realistic within the error-bars and which were completely wrong. We
could then redo the bad point and reach a clean conclusion.

\

This report alternate maths and physics. The odd chapters are about the physical
part of the internship and the even ones about maths. The five first chapter are
about my understanding of prior work that I had to do in order to do what I did.
The last two are about my personal contributions.

In the first chapter I
present the experiment, its modelisation and what we want to prove. In the
second I explain the basics of maximum likelihood reconstruction then I show in
the third chapter how to apply this method for our particular case. In the
fourth chapter I show how to solve the convex optimization problem. Then I can
explain the state of the results before I arrived in chapter 5. In chapter 6 I
explain what I did for computing errorbars for non-linear function and at last
I explain the final results in chapter 7.


\chapter{Experimental setup and goal}

In this chapter I expect a basic knowledge of quantum mechanics, Hilbert space and
Schrodinger equations. In the while report I will use the usual bra-ket notation
and therefore a scalar product linear to the right. The transpose will be
denoted by $A^t$ and the transpose conjugate by $A^*$.
\section{Basic setup description}

Mostly from \cite{Har06}

We want atom interacting with cavities and measuring it at the end

\subsection{Rydberg Ciruclar atoms}

% Rydberg circular atoms
% Cavites

\subsection{Ramsay zone : Interaction with classical field}

\subsection{Cavity resonance and interaction}


\section{Experimental details and calibration}

Mostly from the unknown thesis and other material I have to ask for

\subsection{State preparation}

% oven
% speed filtering Laser with doppler + resonnance
% State filtering (n=52)
% We ignore non-Rydeberg atoms

\subsection{Detection of atom}

% see thesis for functioning of detector

\section{Reconstruction}

Still from \cite{Har06}

\subsection{Ramsay interferometer}

\subsection{Phase shift of QND atoms}

\subsection{What data we have as an output}

Personal





\chapter{Maximum likelihood reconstruction}
\section{Reminders about statistics}

Statistic vocabulary from Wikipédia

\section{Maximum likelihood estimator}

From Wikipédia and Valentin'thesis


\section{General properties}

From Wikipédia and Valentin'thesis and personal work on probability.

\chapter{Effect matrices computation}

\section{Density matrices and Krauss operators}

Anywhere and some \cite{Har06}

Explicitely mention the density matrice as an statistical average of quantum state

\section{Likelihood in terms of Krauss operators}

Mainly \cite{SPRQT16} but some others

\section{Computation of Krauss operators for QND measurement}
Mostly from \cite{VM19} but with a bit of \cite{Har06}

\chapter{Convex optimization}

\section{Reminders on convex optimization}

Aspremont convex optimization course

\section{Projected gradient method}

Ask Rouchon

\section{Convergence criterion}

Ask Rouchon

\chapter{First Results}
\section{Extracting the results and basic error}

Mention the results already proven and put a plot of reconstructed state with errorbars.

We get error on matrix value and photon number

\section{Maxwell Demon experiment}

% Here or first chapter
Mainly Luis notes and presentation


\section{Reminders on Quantum information}

Wikipédia

\subsection{Classical Information theory}

\subsection{Quantum information theory}

\section{Thermodynamical analysis}

Built the new second law from

Mainly the long notes from Luis


\section{Experimental results}

\paragraph{Combination of reconstructions} How we build the matrices
experimentally : from code

Plots and error bars ?

\paragraph{TODO} Need to save the data needed for the report

\chapter{Error estimation and validity proofs}
\section{Quick approach : Monte carlo estimation}
% Only on specific case of probability vectors
\subsection{Reminders on Markov chains}

Continuous Markov chain notes

\subsection{Truncated gaussian simulation}
% Convergence proofs

\section{Probability distribution around maximum}

Full rank and half rank case

Give the formula that we will prove by intuition

Will probably need the transition from the z hessian to the usual notation here.

\section{Exact case}

Use \cite{SPRAL17} but generalise. Will try to compact the proofs

Maybe some stuff will go in the appendices
\subsection{Full rank}

\subsection{Low rank}

\section{Fix the non full span problem}

\paragraph{TODO} original work

\chapter{Final Results}

\section{Approximations and implementation}

Gradient of entropy, projections, ...

Relevant implementation details

Details about error propagation and different approach

\section{Plots}

\paragraph{TODO} Gather data for plots

\section{Interpretation}

Well crafted bullshit

\chapter*{Conclusion} %and bibliography
\addcontentsline{toc}{chapter}{Conclusion}

Conclusion



\vfill

\paragraph{\Huge Thanks}
\addcontentsline{toc}{chapter}{Thanks}

\

\vspace{3mm}

\begin{itemize}

\item Igor Dotsenko
\item Pierre Rouchon
\item Valentin Metillon
\item Luis Najera

\end{itemize}

\vfill


% Bibliography

\addcontentsline{toc}{chapter}{Bibliography}

\bibliographystyle{plain}

{\let\clearpage\relax \bibliography{report}}

\end{document}
